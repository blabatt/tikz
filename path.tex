\section{Path}


\begin{minipage}{6.5cm}\begin{lstlisting}
\path[<option>*]}
   (<from>) -- (<to>) [-{}- (<to>)]*;
\end{lstlisting}\end{minipage}
\textit{(where } \texttt{<from>}, \texttt{<to>} \textit{are coords ... see below)}\\

\textit{Generally, paths start with a \say{move-to} operation (indicated by unpreceeded coord), followed by various \say{path-to} operations. Paths can be drawn, filled, clipped, patterned, and shaded using }\texttt{\textbackslash path[<action>]}\textit{, or with shortcuts to the same like }\texttt{\textbackslash filldraw}\textit{ or }\texttt{\textbackslash shade}\textit{ Some basic examples:}\\

\entry{44mm}{\textbackslash path[draw] (0,0) -{}- (5,2);}{a line} \\
\entry{44mm}{\textbackslash filldraw (-1,5) arc(180:120:1);}{filled arc }\\
\entry{44mm}{\textbackslash shadedraw (210:19mm);}{using polar} \\


%%%%%%%%%%%%%%%%%%%%%%%%%%%%%%%%%%%%%%%%%%
\subsection*{Path Operations}
\textit{$\exists$ various \say{path operations,} exemplified below:}\\
\entry{50mm}{\textbackslash draw (5,0) -{}- (6,1) -{}- (6,0) -{}- cycle;}{line-to} \\
%(a.north) |- (b.west);
\entry{50mm}{\textbackslash draw (a.north) |- (b.west);}{line-to} \\
\entry{50mm}{\textbackslash draw (0,0) -{}- (2,0) .. }{crve-to}\\
\entry{50mm}{\phantom{xx} controls (1,1) and (2,2) .. cycle;}{crve-to} \\
\entry{50mm}{\textbackslash draw (.5,1) rectangle (2,0.5);}{rect.} \\
\entry{50mm}{\textbackslash draw (1,0) circle [radius=5mm];}{circle} \\
\entry{50mm}{\textbackslash draw (1,0) circle }{ellipse} \\
\entry{50mm}{\phantom{xx} [x radius=1cm, y radius=5mm];}{ellipse} \\
\entry{50mm}{\textbackslash draw (8,0) arc [start angle=0,}{arc} \\
\entry{50mm}{\phantom{xx}  end angle=270, x radius=1cm,}{arc} \\
\entry{50mm}{\phantom{xx}  y radius=5mm] -{}- cycle;}{arc} \\
\entry{50mm}{\textbackslash draw (0,0) grid [step=.75cm] (3,2);}{grid} \\
\entry{50mm}{\textbackslash draw (0,0) parabola }{parab.}\\
\entry{50mm}{\phantom{xx} bend (.75,1.75) (1,1.5);}{alt 1} \\
\entry{54mm}{\phantom{xx} [bend pos=0.5] bend +(0,2) +(3,0);}{alt 2} \\
\entry{50mm}{\textbackslash draw (0,0) sin (1,1) cos (2,0);}{sin} \\
\entry{52mm}{\textbackslash draw svg \{M 0 0 L 20 20 h 10 a 10\};}{svg} \\
\entry{50mm}{\textbackslash draw plot coord\textquotesingle s \{(a) (b) (c)\};}{plot} \\
\entry{50mm}{\textbackslash draw (a) to [out=135,in=45] (b);}{to-path} \\
\textit{(Labelling }\texttt{to-path}\textit{s allow them to be styled.)}\\
\entry{50mm}{\textbackslash draw (0,0) foreach \textbackslash x  in \{1,...,3\} }{foreach}\\
\entry{50mm}{\phantom{xx} \{ -{}- (\textbackslash x,1) -{}- (\textbackslash x,0) \};}{foreach} \\
\entry{50mm}{\textbackslash draw let \textbackslash p\{foo\} = (1,1), \textbackslash p2 = (2,0) }{let}\\
\entry{50mm}{\phantom{xx} in (0,0) -{}- (\textbackslash p2) -{}- (\textbackslash p \{foo\});}{let} \\
\entry{50mm}{\textbackslash draw (0,0) to[out=90,in=180]}{node} \\
\entry{50mm}{\phantom{xx} node [sloped,above] {x} (3,2);}{node} \\
\entry{50mm}{\textbackslash draw (1,1) -{}- (2,2) pic \{seagull\};}{pic} \\
\textit{(}\texttt{pic}\textit{s can be coded, path-positioned, \& animated)}\\
\entry{50mm}{\textbackslash draw :xshift = }{animt\textquotesingle n}\\
\entry{51mm}{\phantom{xx} \{0s=\textquotedbl 0cm\textquotedbl, 30s=\textquotedbl -3cm\textquotedbl , repeats\}}{\textquotesingle\textquotesingle\textquotesingle} \\
\entry{51mm}{\phantom{xx} (0,0) circle (5mm);}{\textquotesingle\textquotesingle\textquotesingle} \\


%%%%%%%%%%%%%%%%%%%%%%%%%%%%%%%%%%%%%%%%%%
\subsection*{General Path Options}
{\footnotesize \begin{multicols}{2}\begin{itemize}[leftmargin=1pt,label={}]
    \item name
    \item every path
    \item insert path
    \item rounding corners
    \item rounded corners
    \item sharp corners
    \item color
    \item help lines
\end{itemize}\end{multicols}}


%%%%%%%%%%%%%%%%%%%%%%%%%%%%%%%%%%%%%%%%%%
\subsection*{Draw Options}
{\footnotesize \begin{multicols}{2}\begin{itemize}[leftmargin=1pt,label={}]
    \item line width
    \item ultra thin
    \item very thin
    \item semithick
    \item thick
    \item very thick
    \item ultra thick
    \item line cap
    \item line join
    \item dash pattern
    \item dash phase
    \item dash
    \item dash expand off
    \item solid 
    \item dotted
    \item densely dotted
    \item loosely dotted
    \item dashed
    \item densely dashed
    \item loosely dashed
    \item dash dot
    \item densely dash dot
    \item loosely dash dot
    \item dash dot dot
    \item densely dash dot dot
    \item loosely dash dot dot
    \item double
    \item double distance
    \item \textquotesingle\textquotesingle\textquotesingle  betw. line centers
    \item dbl eql sign distance
\end{itemize}\end{multicols}}


%%%%%%%%%%%%%%%%%%%%%%%%%%%%%%%%%%%%%%%%%%
\subsection*{Miscellaneous Path Actions}
\textit{$\exists$ a number of fill and shade options in addition to fill \say{patterns} and \say{shading} types:}\\
\entry{45mm}{\textbackslash pattern[pattern color=white,}{}
\entry{45mm}{\phantom{xx}pattern=bricks] \dots }{pattern}\\
\entry{45mm}{\textbackslash shadedraw [shading=axis] \dots }{shade type}
{\footnotesize \begin{multicols}{2}\begin{itemize}[leftmargin=1pt,label={}]
    \item fill
    \item pattern
    \item pattern color
    \item nonzero rule
    \item even odd rule
    \item path picture
    \item shading
    \item shading angle
\end{itemize}\end{multicols}}

\textit{Manage picture-text interaction by controlling \say{bounding box} with }\texttt{trim},\texttt{trim left},\texttt{trim right}\textit{, or like:}\\
\code{\textbackslash draw[use as bounding box] \dots}\\

\textit{Clip around any kind of path (drawn, filled, etc):}\\
\code{\textbackslash draw[clip] (0,0) circle (1cm)}\\



%%%%%%%%%%%%%%%%%%%%%%%%%%%%%%%%%%%%%%%%%%%%%%%%%
\subsection*{Plot Path Operation}
\textit{A simplified, tikz-native alternative to pgfplots.}\\
\entry{41mm}{\textbackslash draw plot coordinates }{plot on a path}\\
\entry{41mm}{\phantom{xx}\{ (0,0) (1,2) (3,0) ... \}}{simple coords}\\
\entry{41mm}{\textbackslash draw plot[mark=x,smooth] }{options}\\
\entry{41mm}{\phantom{xx}file \{folder/file.table\}}{extern data}\\
\entry{41mm}{\textbackslash draw plot (\textbackslash x, \{sin(\textbackslash x r)\})}{a function}\\
\entry{48mm}{\textbackslash draw [domain=-3:3,variable=\textbackslash t] }{paramet.}\\
\entry{41mm}{\phantom{xx}plot(\{\textbackslash sin(\textbackslash t r)\},\{cos(\textbackslash t r)\})}{circle}\\[1mm]
\textit{Options:}
{\scriptsize \begin{multicols}{3}\begin{itemize}[leftmargin=1mm,label={}]
    \item variable
    \item samples
    \item domain
    \item samples at
    \item {[x|y]range}
    \item mark
    \item mark repeat
    \item mark phase
    \item mark indices
    \item mark size
    \item smooth
    \item tension
    \item const plot
    \item jump mark
    \item {[x|y]}comb
    \item polar comb
    \item {[x|y]}bar
\end{itemize}\end{multicols}}
