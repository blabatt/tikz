\section{\href{http://mirrors.ctan.org/graphics/pgf/contrib/circuitikz/doc/circuitikzmanual.pdf}{Circuitikz}}

\begin{minipage}{6.5cm}
\begin{lstlisting}
\usepackage{circuitikz}
\begin{circuitikz}[american]
  \draw (<point>) 
  node[<e_type>?,<e_opt>*]?
  [
    to[ <wire_opt>*,
        <e_type>?,
        <e_opt>*   ] 
    node[<e_type>,<e_opt>*]? 
    (<point>)?     
  ]*
\end{lstlisting}
\end{minipage}\ \\

\textit{... where }\texttt{<wire\_opt>}\textit{ $\in$:}\\ 
\begin{itemize}
    \item short
    \item {[*|o]}-[*|o]
    \item i[>\_]?= \% current arrow
    \item f[>\_]?= \% current arrow
    \item l=<name> \% label
    \item v=<volts>
\end{itemize}\ \\

\textit{... and }\texttt{<e\_type>}\textit{ $\in$:}\\ 
\begin{multicols}{2}
\begin{itemize}
    \item R=
    \item V=
    \item C=
    \item L=
    \item rground
    \item switch
    \item isource
    \item vcc
    \item not port
    \item npn
    \item potentiometer
    \item op amp
\end{itemize}
\end{multicols}\ \\

\textit{... which can have names with units like:} \\
\begin{itemize}
    \item <\textbackslash ohm>
    \item <\textbackslash farad>
    \item <\textbackslash siemens>
\end{itemize}\ \\

\textit{\textbf{Note:} can include \textbackslash coord's \& \textbackslash nodes at any place in circuit.  Doing so uses circuit coordinate system.  Many more circuit <e\_type>s available, see \href{https://www.overleaf.com/learn/latex/CircuiTikz_package}{here}; and stylizing options, see \href{http://mirrors.ctan.org/graphics/pgf/contrib/circuitikz/doc/circuitikzmanual.pdf}{documentation}.}

\ \\